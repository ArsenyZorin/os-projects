\documentclass[14pt,a4paper]{report}
\usepackage{cmap}
\usepackage[russian]{babel}
\usepackage{amsmath}
\usepackage{amsfonts}
\usepackage{amssymb}
\usepackage{graphicx}
\usepackage{listings}
\usepackage{hyperref}
\usepackage{float}
\usepackage{fontspec}
\setmainfont{SFNS Display}
\setmonofont{SFNS Display}

\lstset{
	inputencoding=utf8x,
	extendedchars=\true,
	frame=single,
	breaklines=true,
	numbers=left,
	keepspaces = true}

\voffset -24.5mm
\hoffset -5mm
\textwidth 173mm
\textheight 240mm
\oddsidemargin=0mm \evensidemargin=0mm

\author{Зорин А.Г.}
\title{Программа и методика тестирования}
\begin{document}
	\maketitle
	\renewcommand{\thesection}{\arabic{section}}
	\tableofcontents
	\pagebreak
	
\section{Объект тестирования}
Объектом тестирования является программный продукт "Демон звонков". Данное программное обеспечение позволяет отслеживать статус изменения звонков на sim-модуле.Также демон производит первоначальную активацию данного модуля, для обеспечения возможности дальнейшей работы с ним. И получает первоначальную информацию.
		
\section{Цель тестирования}
Целью тестирования является проверка корректности работы программного обеспечения, и проверка реализации заявленных в техническом задании функциональных требований, предъявляемых к данному программному обеспечению.
	
\section{Требования к программе}
Программное обеспечение "Демон звонков" должно:
\begin{itemize}
\item Запускаться при старте системы
\item Производить активацию sim-модуля
\item Получать начальную информацию от модуля
\item Отслеживать изменения статуса звонков
\end{itemize}
		
Методы проверки приведенных выше требований описаны в данном документе.
		
\section{Требования к программной документации}
Для проведения программы испытаний необходимы следующие документы на программное обеспечение:
\begin{itemize}
\item Техническое задание. В нем производится описание требований, которые реализовывает ПО.
\item Текст программы.
\item Руководство системного программиста. В нем описаны способы установки и удаления тестируемого ПО.
\end{itemize}

\section{Методы тестирования}
\begin{table}[H]
\caption{Памятка тестиования}
\label{tabular:timesandtenses}
\begin{center}
\begin{tabular}{| p{0.1\linewidth} | p{0.4\linewidth} | p{0.4\linewidth} |}
\hline
\textbf{Номер метода} & \textbf{Порядок выполнения} & \textbf{Ожидаемый результат проверки} \\
\hline
1 & Если в системе отсутствует oFono --- установить и настроить согласно инструкциям & По завершении полной установки и настройки, в системе должен появиться сервис ofono.service. Для проверки корректности работы ofono рекомендуется воспользоваться тестами, которые лежат в директории \textit{../ofono/tests}.\\
\hline
2 & Компиляция исходных кодов. Для ее выполнения, в директории с исходными кодами, достаточно выполнить команды: 
\begin{verbatim}
mkdir build && cd build
cmake .
make
\end{verbatim} &  Сборка исполняемых файлов должна завершиться без ошибок. В каталоге build должен появиться исполняемый файл calls\_daemon.\\
\hline
3 & Запуск демона  & Во время выполнения данного теста советуется отслеживать изменения в системном логе. Для более удобного отслеживания --- произвести сортировку по демону. В том случае, если демон успешно запущен, не появится никаких сообщений об ошибке или о завершении демона. \\
\hline
4 & Проверка работы демона посредством совершения входящего звонка на модуль & В случае корректной работы демона, в логе отобразится сообщение о создании нового звонка. Также отобразится графическое приложение, уведомляющее о поступлении звонка.\\
\hline
4 & Проверка работы демона посредством ответа на входящий звонок и сброса.  & В случае корректной работы демона, в логе отобразится сообщение об удалении звонка после его завершения. И графическое приложение будет закрыто.\\
\hline
5 & Проверка работы демона посредством совершения исходящего звонка & Когда номер только набран, ничего не будет происходить. Как только вызываемый абонент ответил на звонок --- в лог записывается создание звонка. Графическое приложение изменяется. \\
\hline
\end{tabular}
\end{center}
\end{table}
		
\end{document}
